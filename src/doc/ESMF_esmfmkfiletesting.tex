% $Id$

\subsection{Validating an existing ESMF installation}
\label{esmfmkfiletesting}

It is becoming increasingly common to find pre-installed ESMF libraries on 
professionally maintained HPC systems. Often multiple versions of ESMF are
available via environment modules. Before using such a third-party ESMF 
installation, a user may want ot validate that it is working correctly. System
administrators also often need a simple method to re-validate an existing
ESMF installation, e.g. after a system update. ESMF offers a simple way to
build and run the full regression suite against an existing installation.

A second ESMF source tree is used to run full regression tests against an
existing ESMF installation. To support this, the second source tree must be
of the exact same version as the ESMF installation to be tested. The two
critical environment variables used are {\tt ESMF\_TESTESMFMKFILE}, and 
{\tt ESMFMKFILE}. The following bullets outline the procedure:

\begin{itemize}

\item Check out the same version of ESMF as the installation to be validated.

\item Change into the root directory of the checked out directory tree, and
      set the {\tt ESMF\_DIR} environment variable to the current working
      directory.

\item Set the {\tt ESMFMKFILE} environment variable to point to the 
      {\tt esmf.mk} file of the installation to be validated. If the ESMF
      installation is available via the {\tt module} command, {\tt ESMFMKFILE} 
      will typically be set when loading the module.
      
\item Set the {\tt ESMF\_TESTESMFMKFILE} environment variable to {\tt ON}.

\item Set the {\tt ESMF\_COMPILER}, {\tt ESMF\_COMM}, and {\tt ESMF\_BOPT}
      environment variables to match the values from the {\tt esmf.mk} file.
      
\item Make sure the build environment is set up properly to match the 
      ESMF installation to be validated. On systems using the {\tt module}
      command, this means loading the correct modules.
      
\end{itemize}

At this point all of the test targets discussed in sections \ref{testing} and
\ref{examples} are available. The build targets use the test
and example sources under the local (secondary) source tree, but compile and
link against the ESMF library pointed to by {\tt ESMFMKFILE}. A fully 
functional installation is expected to pass all regression tests.
