% $Id$

%\section{Building and Installing the ESMF}
\subsection{ESMF Download Options}

Major releases of the ESMF software can be downloaded by following
the instructions on the the {\bf Download} link on the ESMF
website, \htmladdnormallink{http://www.earthsystemmodeling.org}{http://www.earthsystemmodeling.org}.

The ESMF is distributed as a full source code tree.
Follow the instructions in the following sections
to build the library and link it with your application.

\subsection{Acquiring Development Snapshots}
Occasionally, it is helpful to acquire a development snapshot of ESMF
in order to test emerging capabilities, optimizations, and bug fixes
before they are available in a formal release.  Development snaphots
are ``use at your own risk.'' Efforts are made to ensure that most unit
and system tests are passing on typical platforms, but there are no
guarantees of the stability of development snapshots. New APIs available
in development snapshots may change before the next release.

Users aware of these risks may check out development snapshots
using the appropriate git tag.

Starting with ESMF 8.3.0 beta snapshot 07, the naming convention for development tags has the form:

\begin{verbatim}
v<VERSION>b<NUMBER>
\end{verbatim}

For example:
\begin{verbatim}
v8.3.0b07
\end{verbatim}

Prior to this version, the tag naming convention for development tags is:

\begin{verbatim}
ESMF_<VERSION>_beta_snapshot_<NUMBER>
\end{verbatim}

For example:
\begin{verbatim}
ESMF_8_2_0_beta_snapshot_23
\end{verbatim}

Use the following example command as a guide to check out a specific development tag:

\begin{verbatim}
  git clone https://github.com/esmf-org/esmf.git --branch v8.3.0b13 --depth 1
\end{verbatim}

Once downloaded, development snapshots are built in the same way as releases.

\subsection{System Requirements}
\label{sec:systemreq}
% $Id$


The following compilers and utilities are required for compiling, linking and
testing the ESMF software. It is good common practice to use a consistent set
of Fortran/C++/C compilers from the same vendor, e.g. GNU, Intel, etc.
However, some vendor {\em combinations} of Fortran, C++, and C compilers,
e.g. Intel ifort with GNU g++, are also supported.
\begin{itemize}
\item Fortran compiler:
  \begin{itemize}
  \item GNU's gfortran v7.0 and newer, or
  \item Intel's ifort v18.0 and newer, or
  \item PGI's pgf90 v18.1 and newer, or
  \item NVHPC's nvfortran, or
  \item NAG's nagfor, or
  \item IBM's xlf, or
  \item CCE's ftn.
  \end{itemize}
\item C++ compiler:
  \begin{itemize}
  \item GNU's g++ v7.0 and newer, or
  \item Intel's icpc v18.0 and newer, or
  \item PGI's pgCC v18.1 and newer, or
  \item NVHPC's nvc++, or
  \item IBM's xlC, or
  \item CCE's CC.
  \item LLVM's clang
  \end{itemize}
\item C compiler:
  \begin{itemize}
  \item GNU's gcc v7.0 and newer, or
  \item Intel's icc v18.0 and newer, or
  \item PGI pgcc v18.1 and newer, or
  \item NVHPC's nvcc, or
  \item IBM's xlc, or
  \item CCE's cc.
  \item LLVM's clang
  \end{itemize}
\item MPI implementation compatible with the above compilers (but also see below
for the MPI-bypass build option):
  \begin{itemize}
  \item OpenMPI v3.0 and newer, or
  \item MPICH v2.1 and newer, or
  \item MVAPICH2 v2.0 and newer, or
  \item IntelMPI v18.0 and newer, or
  \item MPT 2.17 and newer, or
  \item CRAY-MPICH v7.7 and newer.
  \end{itemize}
\item GNU's \htmladdnormallink{gcc compiler}{http://gcc.gnu.org} -
for a standard cpp preprocessor implementation.
\item \htmladdnormallink{GNU Make}{http://www.gnu.org/software/make/make.html}.
\item \htmladdnormallink{Perl}{http://www.perl.com/download.csp} - for running
test scripts.
\end{itemize}

Internal packages that can optionally reference external libraries:
\begin{itemize}
\item LAPACK - version 3.x or newer
\item ParallelIO (PIO) - version 2.5.7 or newer
\item yaml-cpp - tag yaml-cpp-0.6.2 or newer
\end{itemize}

Optional external packages that must be specified for certain functions:
\begin{itemize}
\item NetCDF - version 3.6.x or newer (version 4.4 or newer required by PIO)
\item parallel-NetCDF - version 1.2.0 or newer (version 1.12 or newer required by PIO)
\item Xerces - version 3.1.0 or newer
\end{itemize}

ESMF can be built using a single-processor MPI-bypass library
that comes with ESMF by setting {\tt ESMF\_COMM=mpiuni}. This allows ESMF applications
to be linked and run in single-process mode.

In order to build html and pdf versions of the ESMF documentation, 
\htmladdnormallink{\LaTeX}{http://www.latex-project.org},
the \htmladdnormallink{latex2html}{http://www.latex2html.org}
conversion utility, and the Unix/Linux {\tt dvipdf} utility must be installed.
The csh shell is also required to complete the documentation build.


\subsection{Third Party Libraries}
\label{sec:ThirdParty}

Some portions of the ESMF library can offer enhanced capabilities when
certain third party libraries are available. This section describes
these dependencies and the associated environment settings
that allow the user to control them.

On many platforms, the ESMF library is also created as a shared library.
When third party libraries are called from ESMF, it is recommended that they are
also available as shared libraries.  In cases where they are not, they should at
least be compiled with the position independent code option enabled (e.g., -fPIC on
Linux with gfortran/gcc) where necessary, so that the ESMF shared library
build can successfully incorporate them.

\subsubsection{LAPACK}
\label{sec:lapack}
The patch recovery regridding method of the ESMF Mesh class requires solving
local least squares problems. It uses the
\htmladdnormallink{LAPACK}{http://www.netlib.org/lapack} {\it DGELSY} solver
to carry out this task.

The following environment variables control whether a minimal set of
LAPACK code that comes with ESMF is used, or whether ESMF should link against
an externally available LAPACK installation. Alternatively, ESMF's
LAPACK-dependent features can be turned off altogether.

\begin{description}

\item[ESMF\_LAPACK] Possible value: {\tt "internal"} (default), {\tt "OFF"},
 {\tt "system"}, {\tt "mkl"}, {\tt "netlib"}, {\tt "scsl"}, {\tt openblas}, {\it <userstring>}.

\begin{description}
\item[{\tt "internal"} (default)] ESMF will be compiled with LAPACK-dependent
features. A minimal set of LAPACK/BLAS code included in ESMF will be used
to satisfy the dependencies.

\item[{\tt "OFF"}] Disables LAPACK-dependent code.

\item[{\tt "system"}] A system-dependent external LAPACK/BLAS installation
is used to satisfy the external dependencies of the LAPACK-dependent ESMF code.
Sets {\tt ESMF\_LAPACK\_LIBS} appropriately.

\item[{\tt "mkl"}] The Intel MKL library is used to satisfy the external
dependencies of the LAPACK-dependent ESMF code. When {\tt ESMF\_COMPILER} is set to
{\tt "intel"}, {\tt ESMF\_LAPACK\_LIBS} is set to {\tt "-mkl"}.  Otherwise {\tt ESMF\_LAPACK\_LIBS}
is set to {\tt "-lmkl\_lapack -lmkl"}, unless it is already defined in the user
environment.

\item[{\tt "netlib"}] The NETLIB library is used to satisfy the external
dependencies of the LAPACK-dependent ESMF code. Sets {\tt ESMF\_LAPACK\_LIBS} to
{\tt "-llapack -lblas"}, unless it is already defined in the user environment.

\item[{\tt "scsl"}] The SCSL library is used to satisfy the external
dependencies of the LAPACK-dependent ESMF code. Sets {\tt ESMF\_LAPACK\_LIBS} to
{\tt "-lscs"}, unless it is already defined in the user environment.

\item[{\tt "openblas"}] The OpenBLAS library is used to satisfy the external
dependencies of the LAPACK-dependent ESMF code. Sets {\tt ESMF\_LAPACK\_LIBS} to
{\tt "-openblas"}, unless it is already defined in the user environment.

\item[{\it <userstring>}] Enables ESMF's LAPACK-dependent code, but does not set
a default for {\tt ESMF\_LAPACK\_LIBS}.  {\tt ESMF\_LAPACK\_LIBS}, and if
required, {\tt ESMF\_LAPACK\_LIBPATH}, must be set explicitly in the user
environment.
\end{description}

\item[ESMF\_LAPACK\_LIBPATH] Typical value: {\tt /usr/local/lib} (no default).

Specifies the path where the LAPACK library is located.

\item[ESMF\_LAPACK\_LIBS] Typical value: {\tt "-llapack -lblas"}
(default is dependent on {\tt ESMF\_LAPACK}).

Specifies the linker directive needed to link the LAPACK library to
the application.  On some systems, the BLAS library must also be included.
\end{description}


\subsubsection{NetCDF}
\label{sec:netcdf}
ESMF provides the ability to read Grid and Mesh data in
\htmladdnormallink{NetCDF}{http://www.unidata.ucar.edu/software/netcdf/} format.

Beginning with NetCDF 4.2, the C and Fortran API libraries are released as separate packages.
To compile ESMF with NetCDF 4.2 and newer releases, the {\tt ESMF\_NETCDF} environment variable
can be set to {\tt "split"}.  The {\tt "split"} option requires the NetCDF C library,
and the NetCDF Fortran API library be installed in the same directory.  As an alternative,
the {\tt "nc-config"} option may be used to automatically determine the include and lib directory
locations.  The {\tt "nc-config"} option supports separate C and Fortran directories.


The following environment variables enable, and specify the name and location
of the desired NetCDF library and associated header files:

\begin{description}

\item[ESMF\_NETCDF] Possible value: {\it not set} (default), {\tt "nc-config"}, {\tt "split"},
{\tt "standard"}, {\it <userstring>}.

\begin{description}
\item[{\it not set} (default)] NetCDF-dependent features will be disabled.
The {\tt ESMF\_NETCDF\_INCLUDE}, {\tt ESMF\_NETCDF\_LIBPATH}, and
{\tt ESMF\_NETCDF\_LIBS} environment variables will be ignored.

\item[{\tt "nc-config"}] The NetCDF {\tt nc-config} and if available, {\tt nf-config},
tools will be used to determine the proper settings of {\tt ESMF\_NETCDF\_INCLUDE},
{\tt ESMF\_NETCDF\_LIBPATH}, and {\tt ESMF\_NETCDF\_LIBS}.  The shell {\tt PATH}
environment variable must include the NetCDF bin directories where {\tt nc-config}
and {\tt nf-config} reside.  This option supports having the main NetCDF library and the
Fortran API library reside in separate directories.

\item[{\tt "split"}] {\tt ESMF\_NETCDF\_LIBS} will be set to
{\tt "-lnetcdff -lnetcdf"}.  This option is useful for systems
which have the Fortran and C bindings archived in separate library files.
The {\tt ESMF\_NETCDF\_INCLUDE} and {\tt ESMF\_NETCDF\_LIBPATH}
environment variables will also be used, if defined.

\item[{\tt "standard"}] {\tt ESMF\_NETCDF\_LIBS} will be set to
{\tt "-lnetcdf"}.  This option is useful when the Fortran and
C bindings are archived together in the same library file.  The {\tt ESMF\_NETCDF\_INCLUDE}
and {\tt ESMF\_NETCDF\_LIBPATH} environment variables will also be used,
if defined.

\item[{\it <userstring>}] If set, {\tt ESMF\_NETCDF\_INCLUDE},
{\tt ESMF\_NETCDF\_LIBPATH}, and {\tt ESMF\_NETCDF\_LIBS} environment
variables will be used, if defined.
\end{description}

\item[ESMF\_NETCDF\_INCLUDE] Typical value: {\tt /usr/local/include}
(no default).

Specifies the path where the NetCDF header files are located.

\item[ESMF\_NETCDF\_LIBPATH] Typical value: {\tt /usr/local/lib} (no default).

Specifies the path where the NetCDF library file is located.

\item[ESMF\_NETCDF\_LIBS] Typical value: {\tt "-lnetcdf"}

Specifies the linker directives needed to link the NetCDF library to
the application.

The default value depends on the setting of {\tt ESMF\_NETCDF}.  For the
typical case where {\tt ESMF\_NETCDF} is set to {\tt "standard"},
{\tt ESMF\_NETCDF\_LIBS} is set to {\tt "-lnetcdf"}.
When {\tt ESMF\_NETCDF} is set to {\tt "split"}, {\tt ESMF\_NETCDF\_LIBS}
is set to {\tt "-lnetcdff -lnetcdf"}.

If the hdf5 library is required, append {\tt "-lhdf5\_hl -lhdf5"} to the
desired setting.  E.g. {\tt "-lnetcdff -lnetcdf -lhdf5\_hl -lhdf5"}
\end{description}

\subsubsection{Parallel-NetCDF}
\label{sec:pnetcdf}
ESMF provides the ability to write data using
\htmladdnormallink{Parallel-NetCDF}{http://trac.mcs.anl.gov/projects/parallel-netcdf}.

Some file systems, for example \htmladdnormallink{Lustre}{http://wiki.lustre.org}, may need
to have locking attributes enabled when the file system is mounted.

The following environment variables enable and specify the name and
location of the desired Parallel-NetCDF library and associated header files:

\begin{description}
\item[ESMF\_PNETCDF] Possible value: {\it not set} (default), {\tt "pnetcdf-config"},
{\tt "standard"}, {\it <userstring>}.

When defined, enables the use of Parallel-NetCDF.

\begin{description}
\begin{sloppypar}
\item[{\it not set} (default)] PNETCDF-dependent features will be disabled.
The {\tt ESMF\_PNETCDF\_INCLUDE}, {\tt ESMF\_PNETCDF\_LIBPATH}, and
{\tt ESMF\_PNETCDF\_LIBS} environment variables will be ignored.
\end{sloppypar}

\item[{\tt "pnetcdf-config"}] The PNetCDF {\tt pnetcdf-config} tool will be used
to determine the proper settings of {\tt ESMF\_PNETCDF\_INCLUDE},
{\tt ESMF\_PNETCDF\_LIBPATH}, and {\tt ESMF\_PNETCDF\_LIBS}.  The shell {\tt PATH}
environment variable must include the PNetCDF bin directory where {\tt pnetcdf-config}
resides.

\item[{\tt "standard"}] {\tt ESMF\_PNETCDF\_LIBS} will be set to
{\tt "-lpnetcdf"}.  The {\tt ESMF\_PNETCDF\_INCLUDE} and
{\tt ESMF\_PNETCDF\_LIBPATH} environment variables will also be used,
if defined.

\item[{\it <userstring>}] If set, {\tt ESMF\_PNETCDF\_INCLUDE},
{\tt ESMF\_PNETCDF\_LIBPATH}, and {\tt ESMF\_PNETCDF\_LIBS} environment
variables will be used.
\end{description}

\item[ESMF\_PNETCDF\_INCLUDE] Typical value: {\tt /usr/local/include}
(no default).

Specifies the path where the Parallel-NetCDF header files are located.

\item[ESMF\_PNETCDF\_LIBPATH] Typical value: {\tt /usr/local/lib} (no default).

Specifies the path where the Parallel-NetCDF library file is located.

\item[ESMF\_PNETCDF\_LIBS] Typical value: {\tt "-lpnetcdf"}.

Specifies the linker directives needed to link the Parallel-NetCDF library to the
application.
\end{description}

\subsubsection{PIO}
\label{sec:pio}
ESMF provides the ability to read and write data in
NetCDF format through \htmladdnormallink{ParallelIO (PIO)}
{https://github.com/NCAR/ParallelIO}, a third-party I/O software
library that is integrated into the ESMF library. The following environment
variable enables PIO functionality inside of ESMF.

The PIO code depends on MPI I/O support by the underlying MPI
implementation for parallel I/O. Almost all current MPI
implementations support MPI I/O to the required degree. For NetCDF format
support the integrated PIO code depends on {\tt ESMF\_NETCDF} (see \ref{sec:netcdf})
being enabled and optionally {\tt ESMF\_PNETCDF} (see \ref{sec:pnetcdf})
being enabled.

\begin{description}
\item[ESMF\_PIO] Possible value: {\it not set} (default), {\tt "internal"},
{\tt "external"}, {\tt "OFF"}.

\begin{description}
\item[{\it not set} (default)] PIO-dependent features will be enabled on supported
platforms, as determined by the ESMF build configuration.

\item[{\tt "internal"}] PIO-dependent features will be enabled and will use the
PIO library that is included and built with ESMF. Internal builds of PIO require
CMake version 2.8.12 or newer be available in the path.

\item[{\tt "external"}] PIO-dependent features will be enabled and will use an
external PIO library.  The additional parameters {\tt ESMF\_PIO\_INCLUDE} 
(path to PIO include files) and {\tt ESMF\_PIO\_LIBPATH} (path to PIO library files)
should also be set when using this option. The minimum version of PIO for
this option is 2.5.7. 

\item[{\tt "OFF"}] Disables PIO-dependent code.

\end{description}


\item[ESMF\_PIO\_INCLUDE] (no default)

Specifies the path where the PIO header files are located.

\item[ESMF\_PIO\_LIBPATH] (no default)

Specifies the path where the PIO library is located.

\end{description}

\subsubsection{Accelerator Software Stacks}
\label{sec:acc}
ESMF provides the ability to query various third party accelerator software
stacks and gather information about the accelerator devices available in a
system. The users can query the number of accelerator devices accessible
from a PET using the OpenCL, OpenACC, Intel MIC and OpenMP software stacks.

The following environment variables enable, and specify the name and location
of the desired accelerator software stacks and associated header files:

\begin{description}

\item[ESMF\_ACC\_SOFTWARE\_STACK] Possible value:
{\it not set} (default), {\tt "opencl"}, {\tt "openacc"},
{\tt "intelmic"}, {\tt "openmp4"}.

\begin{description}
\item[{\it not set} (default)] All accelerator software stack related features
 will be disabled.
The {\tt ESMF\_ACC\_SOFTWARE\_STACK\_INCLUDE},
{\tt ESMF\_ACC\_SOFTWARE\_STACK\_LIBPATH}, and
{\tt ESMF\_ACC\_SOFTWARE\_STACK\_LIBS} environment variables will be ignored.

\item[{\tt "opencl"}] The ESMF library will use the OpenCL
framework to query information about accelerator devices in the system.
The {\tt ESMF\_ACC\_SOFTWARE\_STACK\_INCLUDE},
{\tt ESMF\_ACC\_SOFTWARE\_STACK\_LIBPATH} and
{\tt ESMF\_ACC\_SOFTWARE\_STACK\_LIBS} environment variables will be used
to build and link the library.

\item[{\tt "openacc"}] The ESMF library will use the interfaces defined
in the OpenACC standard to query information about accelerator devices
in the system.
The {\tt ESMF\_ACC\_SOFTWARE\_STACK\_INCLUDE},
{\tt ESMF\_ACC\_SOFTWARE\_STACK\_LIBPATH} and
{\tt ESMF\_ACC\_SOFTWARE\_STACK\_LIBS} environment variables are not typically
defined since the standard is supported inherently by a OpenACC standard
compliant compiler.

\item[{\tt "intelmic"}] The ESMF library will use the interfaces defined
by the Intel MIC software stack to query information about accelerator devices
in the system.
The {\tt ESMF\_ACC\_SOFTWARE\_STACK\_INCLUDE},
{\tt ESMF\_ACC\_SOFTWARE\_STACK\_LIBPATH} and
{\tt ESMF\_ACC\_SOFTWARE\_STACK\_LIBS} environment variables are not typically
defined since the standard is supported inherently by the Intel compiler.

\item[{\tt "openmp4"}] The ESMF library will use the interfaces defined
in the OpenMP v4.0 standard to query information about accelerator devices
in the system.
The {\tt ESMF\_ACC\_SOFTWARE\_STACK\_INCLUDE},
{\tt ESMF\_ACC\_SOFTWARE\_STACK\_LIBPATH} and
{\tt ESMF\_ACC\_SOFTWARE\_STACK\_LIBS} environment variables are not typically
defined since the standard is supported inherently by a standard compliant
compiler.

\end{description}

\item[ESMF\_ACC\_SOFTWARE\_STACK\_INCLUDE] (no default)

Specifies the path where the header files for the accelerator software
stack is located. If not set, this environment variable is ignored.

\item[ESMF\_ACC\_SOFTWARE\_STACK\_LIBPATH] (no default)

Specifies the path where the libraries for the accelerator software
stack is located. If not set, this environment variable is ignored.

\item[ESMF\_ACC\_SOFTWARE\_STACK\_LIBS] (no default)

Specifies the linker directives required to link the library with
the accelerator software stack. If not set, this environment variable
is ignored.

\end{description}


\subsubsection{XERCES}
\label{sec:xerces}
ESMF provides the ability to read Attribute data in XML file format
via the \htmladdnormallink{XERCES C++}{http://xerces.apache.org/xerces-c/}
library.  (Writing Attribute XML files is performed with the standard C++
output file stream facility, rather than with Xerces).  The following
environment variables enable, and specify the name and location of the
desired XERCES C++ library and associated header files:

\begin{description}

\item[ESMF\_XERCES] Possible value: {\it not set} (default), {\tt "standard"},
{\it <userstring>}.

\begin{description}
\item[{\it not set} (default)] XERCES-dependent features will be disabled.
The {\tt ESMF\_XERCES\_INCLUDE}, {\tt ESMF\_XERCES\_LIBPATH}, and
{\tt ESMF\_XERCES\_LIBS} environment variables will be ignored.

\item[{\tt "standard"}] {\tt ESMF\_XERCES\_LIBS} will be set to
{\tt "-lxerces-c"}.  The {\tt ESMF\_XERCES\_INCLUDE} and
{\tt ESMF\_XERCES\_LIBPATH} environment variables will also be used,
if defined.

\item[{\it <userstring>}] If set, {\tt ESMF\_XERCES\_INCLUDE},
{\tt ESMF\_XERCES\_LIBPATH}, and {\tt ESMF\_XERCES\_LIBS} environment
variables will be used, if defined.
\end{description}

\item[ESMF\_XERCES\_INCLUDE] Typical value: {\tt /usr/local/include}
(no default).

Specifies the path where the XERCES C++ header files are located.

\item[ESMF\_XERCES\_LIBPATH] Typical value: {\tt /usr/local/lib} (no default).

Specifies the path where the XERCES C++ library file is located.

\item[ESMF\_XERCES\_LIBS] Typical value: {\tt "-lxerces-c"}.

Specifies the linker directives needed to link the XERCES C++ library to
the application.

The default value depends on the setting of {\tt ESMF\_XERCES}.  For the
typical case where {\tt ESMF\_XERCES} is set to {\tt "standard"},
{\tt ESMF\_XERCES\_LIBS} is set to {\tt "-lxerces-c"}.
\end{description}


\subsubsection{yaml-cpp}
\label{sec:yaml-cpp}
Support for I/O in YAML Ain't Markup Language
(\htmladdnormallink{YAML\texttrademark}{http://yaml.org})
may be added to ESMF through the open-source
\htmladdnormallink{yaml-cpp}{https://github.com/jbeder/yaml-cpp} library,
a YAML parser and emitter written in C++ that implements
\htmladdnormallink{YAML Version 1.2 specifications}{http://yaml.org/spec/1.2/spec.html}.

ESMF includes the option to build the yaml-cpp from sources kept inside the
ESMF source tree, or to link against an external build of the yaml-cpp library.
The following environment variables control the details of how ESMF interacts
with yaml-cpp:

\begin{description}

\item[ESMF\_YAMLCPP] Possible values: {\it "internal"} (default), {\tt "OFF"},
{\tt "standard"}, {\it <userstring>}.

\begin{description}
\item[{\it "internal"} (default)] The YAML-dependent code inside of ESMF will
be enabled.
The yaml-cpp sources included with ESMF will be used to provide YAML support.
The {\tt ESMF\_YAMLCPP\_INCLUDE}, {\tt ESMF\_YAMLCPP\_LIBPATH}, and
{\tt ESMF\_YAMLCPP\_LIBS} environment variables will be ignored.

\item[{\tt "OFF"}] Disables YAML-dependent code.

\item[{\tt "standard"}] The YAML-dependent code inside of ESMF will
be enabled.
{\tt ESMF\_YAMLCPP\_LIBS} will be set to {\tt "-lyaml-cpp"} if not set.
The {\tt ESMF\_YAMLCPP\_INCLUDE} and {\tt ESMF\_YAMLCPP\_LIBPATH} environment
variables will also be used, if defined.

\item[{\it <userstring>}] The YAML-dependent code inside of ESMF will
be enabled.
If set, {\tt ESMF\_YAMLCPP\_INCLUDE}, {\tt ESMF\_YAMLCPP\_LIBPATH}, and
{\tt ESMF\_YAMLCPP\_LIBS} environment variables will be used.
\end{description}

\item[ESMF\_YAMLCPP\_INCLUDE] Typical value: {\tt /usr/local/include}
(no default).

Specifies the path where the yaml-cpp C++ header files are located.

\item[ESMF\_YAMLCPP\_LIBPATH] Typical value: {\tt /usr/local/lib} (no default).

Specifies the path where the yaml-cpp C++ library file is located.

\item[ESMF\_YAMLCPP\_LIBS] Typical value: {\tt "-lyaml-cpp"}.

Specifies the linker directives needed to link the yaml-cpp C++ library to
the application.

The default value depends on the setting of {\tt ESMF\_YAMLCPP}.  For the
typical case where {\tt ESMF\_YAMLCPP} is set to {\tt "standard"},
{\tt ESMF\_YAMLCPP\_LIBS} is set to {\tt "-lyaml-cpp"}.
\end{description}


\subsubsection{MOAB}
\label{sec:MOAB}

The Mesh Oriented datABase
(\htmladdnormallink{MOAB}{https://sigma.mcs.anl.gov/moab-library/})
can be used to build an ESMF unstructured Mesh as an alternative to the 
"native" ESMF Mesh implementation. The decision to use either MOAB or the 
native ESMF Mesh implementation is made at run time. This aspect is described 
in the Reference Manual, section {\tt ESMF\_MeshSetMOAB()} .The default is to use 
the native ESMF Mesh. ESMF will build an internal version of MOAB by default, 
but an external MOAB installation can be used if desired. The build parameters 
covered in this section are used to determine which version of MOAB is 
available to ESMF. 

\begin{description}

\item[ESMF\_MOAB] Possible values: {\tt "internal"} (default), {\tt "OFF"},
{\tt "external"}.

\begin{description}
\item[{\tt "internal"} (default)] The MOAB dependent code inside of ESMF will
be enabled.
The MOAB sources included with ESMF will be used to provide MOAB support.
The {\tt ESMF\_MOAB\_INCLUDE}, {\tt ESMF\_MOAB\_LIBPATH}, and
{\tt ESMF\_MOAB\_LIBS} environment variables will be ignored.

\item[{\tt "OFF"}] Disables MOAB dependent code.

\item[{\tt "external"}] The MOAB dependent code inside of ESMF will
be enabled.
The {\tt ESMF\_MOAB\_INCLUDE}, {\tt ESMF\_MOAB\_LIBPATH} and 
{\tt ESMF\_MOAB\_LIBS} environment variables must also be specified.
\end{description}

\item[ESMF\_MOAB\_INCLUDE] Typical value: {\tt /usr/local/include}
(no default).

Specifies the path where the MOAB C++ header files are located.

\item[ESMF\_MOAB\_LIBPATH] Typical value: {\tt /usr/local/lib} (no default).

Specifies the path where the MOAB C++ library file is located.

\item[ESMF\_MOAB\_LIBS] Typical value: {\tt "-lMOAB"} (no default).

Specifies the linker directives needed to link the MOAB C++ library to
the application.

\end{description}

\subsection{ESMF Environment Variables}
\label{EnvironmentVariables}

The following is a full alphabetical list of all environment variables which
are used by the ESMF build system. In many cases only {\tt ESMF\_DIR} must be
set. On Linux and Darwin systems {\tt ESMF\_COMPILER} and {\tt ESMF\_COMM} must
also be set to select the appropriate Fortran and C++ compilers and MPI
implementation. The other variables have default values which work for
most systems.

\begin{description}

\item[ESMF\_ABI]
Possible value: {\tt 32}, {\tt 64}, {\tt x86\_64\_32}, {\tt x86\_64\_small}, {\tt x86\_64\_medium}

If a system supports 32-bit and 64-bit (pointer wordsize) application binary
interfaces (ABIs), this variable can be set to select which ABI to use. Valid
values are {\tt 32} or {\tt 64}. By default the most common ABI is chosen. On
x86\_64 architectures three additional, more specific ABI settings are available,
{\tt x86\_64\_32}, {\tt x86\_64\_small} and {\tt x86\_64\_medium}.

\item[ESMF\_ARRAY\_LITE]
Possible value: {\tt TRUE}, {\tt FALSE} (default)

Not normally set by user. ESMF auto-generates subroutine interfaces for a wide
variety of data arrays of different ranks, shapes, and types. Setting this
variable to {\tt TRUE} instructs ESMF to {\em not} generating interfaces for
5D, 6D, and 7D arrays. This shrinks the amount of autogenerated code as well
as the number of overloaded interfaces.

\item[ESMF\_BOPT]
Possible value: {\tt g}, {\tt O} (default)

This environment variable controls the build option. To make a debuggable
version of the library set {\tt ESMF\_BOPT} to {\tt g} before building. The
default is {\tt O} (capital oh) which builds an optimized version of the
library. If {\tt ESMF\_BOPT} is {\tt O}, {\tt ESMF\_OPTLEVEL} can also be set
to a numeric value between 0 and 4 to select a specific optimization level.

\item[ESMF\_COMM]
Possible value: {\em system-dependent}

On systems with a vendor-supplied MPI communications library, the vendor library
is chosen by default for communications. On these systems {\tt ESMF\_COMM} is
set to {\tt mpi}, signaling to the ESMF build system to use the vendor MPI
implementation.
For other systems (e.g. Linux or Darwin) where a multitude of MPI
implementations are available, {\tt ESMF\_COMM} must be set to indicate which
implementation is used to build the ESMF library. Set {\tt ESMF\_COMM} according
to your situation to: {\tt mpt, mpich, mpich1, mpich2, mpich3, mvapich2, lam, openmpi}
or {\tt intelmpi}. {\tt ESMF\_COMM} may also be set to {\tt user} indicating
that the user will set all the required flags using advanced ESMF environment
variables.  Some individual MPI builds may create additional libraries that
need to be linked in, such as the legacy C++ bindings. These may be specified
via the {\tt ESMF\_CXXLINKLIBS} and {\tt ESMF\_F90LINKLIBS} environment
variables.

Alternatively, ESMF comes with a single-processor MPI-bypass library which is
the default for Linux and Darwin systems. To force the use of this bypass
library set {\tt ESMF\_COMM} equal to {\tt mpiuni}.

\item[ESMF\_COMPILER]
Possible value: {\em system-dependent}

The ESMF library build requires a working Fortran90 and C++ compiler. On
platforms that don't come with a single vendor supplied compiler suite
(e.g. Linux or Darwin) {\tt ESMF\_COMPILER} must be set to select which Fortran
and C++ compilers are being used to build the ESMF library. Notice that setting
the {\tt ESMF\_COMPILER} variable does {\em not} affect how the compiler
executables are located on the system. {\tt ESMF\_COMPILER} (together with
{\tt ESMF\_COMM}) affect the name that is expected for the compiler executables.
Furthermore, the {\tt ESMF\_COMPILER} setting is used to select compiler and
linker flags consistent with the compilers indicated.

By default Fortran and C++ compiler executables are expected to be located in
a location contained in the user's {\tt PATH} environment variable. This means
that if you cannot locate the correct compiler executable via the {\tt which}
command on the shell prompt the ESMF build system won't find it either!

There are advanced ESMF environment variables that can be used to select
specific compiler executables by specifying the full path. This can be used to
pick specific compiler executables without having to modify the {\tt PATH}
environment variable.

Use 'gmake info' to see which compiler executables the ESMF build system will
be using according to your environment variable settings.

To see possible values for {\tt ESMF\_COMPILER}, cd to
{\tt \$ESMF\_DIR/build\_config} and list the directories there. The first part
of each directory name corresponds to the output of 'uname -s' for this
platform. The second part contains possible values for {\tt ESMF\_COMPILER}. In
some cases multiple combinations of Fortran and C++ compilers are possible, e.g.
there is {\tt intel} and {\tt intelgcc} available for Linux. Setting
{\tt ESMF\_COMPILER} to {\tt intel} indicates that both Intel Fortran and
C++ compilers are used, whereas {\tt intelgcc} indicates that the Intel Fortran
compiler is used in combination with GCC's C++ compiler.

If you do not find a configuration that matches your situation you will need to
port ESMF.

\item[ESMF\_CXX]
Possible value: {\em executable}

This variable can be used to override the default C++ compiler and linker
front-end executables. The executable may be specified with absolute path
overriding the location determined by default from the user's PATH variable.

\item[ESMF\_CXXCOMPILEOPTS]
Possible value: {\em list of flags}

Prepend compiler flags to the list of flags the ESMF build system determines.

\item[ESMF\_CXXCOMPILEPATHS]
Possible value: {\em list of paths, each prepended with -I}

Prepend compiler search paths to the list of search paths the ESMF build system
determines.

\item[ESMF\_CXXCOMPILER]
Possible value: {\em executable}

This variable can be used to override the default C++ compiler
front-end executables. The executable may be specified with absolute path
overriding the location determined by default from the user's PATH variable.

\item[ESMF\_CXXLINKER]
Possible value: {\em executable}

This variable can be used to override the default C++ linker
front-end executables. The executable may be specified with absolute path
overriding the location determined by default from the user's PATH variable.

\item[ESMF\_CXXLINKLIBS]
Possible value: {\em list of libraries, each prepended with -l}

Prepend libraries to the list of libraries the ESMF build system determines.

\item[ESMF\_CXXLINKOPTS]
Possible value: {\em list of flags}

Prepend linker flags to the list of flags the ESMF build system determines.

\item[ESMF\_CXXLINKPATHS]
Possible value: {\em list of paths, each prepended with -L}

Prepend linker search paths to the list of search paths the ESMF build system
determines.

\item[ESMF\_CXXLINKRPATHS]
Possible value: {\em list of paths, each prepended with the correct rpath option}

Prepend linker rpaths to the list of rpaths the ESMF build system determines.

\item[ESMF\_CXXOPTFLAG]
Possible value: {\em flag}

This variable can be used to override the default C++ optimization flag.

\item[ESMF\_CXXSTD]
Possible value: {\em integer} or {\em default} or {\em sysdefault}

Used to set the C++ language standard. If unset or {\em default}, the ESMF default C++ language standard is used: C++11.
If set to an integer, the integer is used to indicate the respective C++ language standard to the compiler. ESMF does not check whether the integer corresponds to an existing language standard.
Setting {\em sysdefault} results in usage of the compiler specific default C++ language standard. This can lead to build issue if the compiler default is below the level required by ESMF.

\item[ESMF\_DEFER\_LIB\_BUILD]
Possible value: {\em ON} (default), {\em OFF}

This variable can be used to override the deferring of the build of the
ESMF library.  By default, the library is built after all of the source
files have been compiled.  This speeds up the build process. It also
allows parallel compilation of source code when the -j flag is used with
gmake.  Setting this environment variable to {\tt OFF} forces the library to
be updated after each individual compilation, thus disabling the ability
to use parallel compilation.

\item[ESMF\_DIR]
Possible value: {\em absolute path}

The environment variable {\tt ESMF\_DIR} must be set to the full pathname
of the top level ESMF directory before building the framework. This is the
only environment variable which is required to be set on all platforms under
all conditions.

\item[ESMF\_F90]
Possible value: {\em executable}

This variable can be used to override the default Fortran90 compiler and linker
front-end executables. The executable may be specified with absolute path
overriding the location determined by default from the user's PATH variable.

\item[ESMF\_F90COMPILEOPTS]
Possible value: {\em list of flags}

Prepend compiler flags to the list of flags the ESMF build system determines.

\item[ESMF\_F90COMPILEPATHS]
Possible value: {\em list of paths, each prepended with -I}

Prepend compiler search paths to the list of search paths the ESMF build system
determines.

\item[ESMF\_F90COMPILER]
Possible value: {\em executable}

This variable can be used to override the default Fortran90 compiler
front-end executables. The executable may be specified with absolute path
overriding the location determined by default from the user's PATH variable.

\item[ESMF\_F90IMOD]
Possible value: {\em flag}

This variable can be used to override the default flag (-I) used to specify a
Fortran module directory.

\item[ESMF\_F90LINKER]
Possible value: {\em executable}

This variable can be used to override the default Fortran90 linker
front-end executables. The executable may be specified with absolute path
overriding the location determined by default from the user's PATH variable.

\item[ESMF\_F90LINKLIBS]
Possible value: {\em list of libraries, each prepended with -l}

Prepend libraries to the list of libraries the ESMF build system determines.

\item[ESMF\_F90LINKOPTS]
Possible value: {\em list of flags}

Prepend linker flags to the list of flags the ESMF build system determines.

\item[ESMF\_F90LINKPATHS]
Possible value: {\em list of paths, each prepended with -L}

Prepend linker search paths to the list of search paths the ESMF build system
determines.

\item[ESMF\_F90LINKRPATHS]
Possible value: {\em list of paths, each prepended with the correct rpath option}

Prepend linker rpaths to the list of rpaths the ESMF build system determines.

\item[ESMF\_F90OPTFLAG]
Possible value: {\em flag}

This variable can be used to override the default  Fortran90 optimization flag.

\item[ESMF\_INSTALL\_BINDIR]
Possible value: {\em relative or absolute path}

Location into which to install the ESMF apps during installation. This
location can be specified as absolute path (starting with "/") or relative to
{\tt ESMF\_INSTALL\_PREFIX}.

\item[ESMF\_INSTALL\_DOCDIR]
Possible value: {\em relative or absolute path}

Location into which to install the documentation during installation. This
location can be specified as absolute path (starting with "/") or relative to
{\tt ESMF\_INSTALL\_PREFIX}.

\item[ESMF\_INSTALL\_HEADERDIR]
Possible value: {\em relative or absolute path}

Location into which to install the header files during installation. This
location can be specified as absolute path (starting with "/") or relative to
{\tt ESMF\_INSTALL\_PREFIX}.

\item[ESMF\_INSTALL\_LIBDIR]
Possible value: {\em relative or absolute path}

Location into which to install the library files during installation. This
location can be specified as absolute path (starting with "/") or relative to
{\tt ESMF\_INSTALL\_PREFIX}.

\item[ESMF\_INSTALL\_MODDIR]
Possible value: {\em relative or absolute path}

Location into which to install the F90 module files during installation. This
location can be specified as absolute path (starting with "/") or relative to
{\tt ESMF\_INSTALL\_PREFIX}.

\item[ESMF\_INSTALL\_PREFIX]
Possible value: {\em relative or absolute path}

This variable specifies the prefix of the installation path used during the
installation process accessible thought the install target. Libraries, F90
module files, header files and documentation all are installed relative to
{\tt ESMF\_INSTALL\_PREFIX} by default. The {\tt ESMF\_INSTALL\_PREFIX} may be
provided as absolute path (starting with "/") or relative to {\tt ESMF\_DIR}.

\item[ESMF\_LAPACK]
See \ref{sec:lapack}

\item[ESMF\_LAPACK\_LIBPATH]
See \ref{sec:lapack}

\item[ESMF\_LAPACK\_LIBS]
See \ref{sec:lapack}

\item[ESMF\_MACHINE]
Possible value: output of {\tt uname -m} where available.

Not normally set by user. This variable indicates architectural details about
the machine on which the ESMF library is being built. The value of this
variable will affect which ABI settings are available and what they mean.
{\tt ESMF\_MACHINE} is set automatically.

\item[ESMF\_MPIBATCHOPTIONS]
Possible value: {\em system-dependent}

Variable used to pass system-specific queue options to the batch system.
Typically the queue, project and limits are set.
See section \ref{ESMFRunSetting} for a discussion of this option.

\item[ESMF\_MPILAUNCHOPTIONS]
Possible value: {\em system-dependent}

Variable used to pass system-specific options to the MPI launch facility.
See section \ref{ESMFRunSetting} for a discussion of this option.

\item[ESMF\_MPIMPMDRUN]
Possible value: {\em executable}

This variable can be used to override the default utility used to launch
parallel execution of ESMF test applications in MPMD mode. The executable in
{\tt ESMF\_MPIMPMDRUN} may be specified with path.

\item[ESMF\_MPIRUN]
Possible value: {\em executable}

This variable can be used to override the default utility used to launch
parallel ESMF test or example applications. The executable in {\tt ESMF\_MPIRUN}
may be specified with path.
See section \ref{ESMFRunSetting} for a discussion of this option.

\item[ESMF\_MPISCRIPTOPTIONS]
Possible value: {\em system-dependent}

Variable used to pass system-specific options to the first level MPI script
accessed by ESMF.
See section \ref{ESMFRunSetting} for a discussion of this option.

\item[ESMF\_NETCDF]
See \ref{sec:netcdf}

\item[ESMF\_NETCDF\_INCLUDE]
See \ref{sec:netcdf}

\item[ESMF\_NETCDF\_LIBPATH]
See \ref{sec:netcdf}

\item[ESMF\_NETCDF\_LIBS]
See \ref{sec:netcdf}

\item[ESMF\_NO\_INTEGER\_1\_BYTE]
Possible value: {\tt TRUE} (default), {\tt FALSE}

Not normally set by user. Setting this variable to {\tt FALSE} instructs
ESMF to generating data array interfaces for data types of 1-byte integers.

\item[ESMF\_NO\_INTEGER\_2\_BYTE]
Possible value: {\tt TRUE} (default), {\tt FALSE}

Not normally set by user. Setting this variable to {\tt FALSE} instructs
ESMF to generating data array interfaces for data types of 2-byte integers.

\item[ESMF\_OPENACC]
Possible value: {\tt ON}, {\tt OFF} (default)

Compiles and links the ESMF library with OpenACC compiler flags.

\item[ESMF\_OPENMP]
Possible value: {\tt ON}, {\tt OFF} (default is system dependent)

Compiles and links the ESMF library with OpenMP compiler flags.

\item[ESMF\_OPTLEVEL]
Possible value: {\em numerical value}

See {\tt ESMF\_BOPT} for details.

\item[ESMF\_OS]
Possible value: output of {\tt uname -s} except when cross-compiling or for
{\tt UNICOS/mp} where {\tt ESMF\_OS} is {\tt Unicos}.

Not normally set by user unless cross-compiling. This variable indicates the
target system for which the ESMF library is being built. Under normal
circumstances, i.e. ESMF is being build on the target system, {\tt ESMF\_OS} is
set automatically. However, when cross-compiling for a different target system
{\tt ESMF\_OS} must be set to the respective target OS. For example, when
compiling for the Cray X1 on an interactive X1 node {\tt ESMF\_OS} will be set
automatically. However, when ESMF is being cross-compiled for the X1 on a Linux
host the user must set {\tt ESMF\_OS} to {\tt Unicos} manually in order to
indicate the intended target platform.

\item[ESMF\_PNETCDF]
See \ref{sec:pnetcdf}

\item[ESMF\_PNETCDF\_INCLUDE]
See \ref{sec:pnetcdf}

\item[ESMF\_PNETCDF\_LIBPATH]
See \ref{sec:pnetcdf}

\item[ESMF\_PNETCDF\_LIBS]
See \ref{sec:pnetcdf}

\item[ESMF\_PTHREADS]
Possible value: {\tt ON} (default on most platforms), {\tt OFF}

This compile-time option controls ESMF's dependency on a functioning
Pthreads library. The default option is set to {\tt ON} with the exception
of IRIX64 and platforms that don't provide Pthreads. On IRIX64 the use of
Pthreads in ESMF is disabled by default because the Pthreads library conflicts
with the use of OpenMP on this platform.

The user can override the default setting of {\tt ESMF\_PTHREADS} on all
platforms that provide Pthread support. Setting the {\tt ESMF\_PTHREADS}
environment variable to {\tt OFF} will disable ESMF's Pthreads feature set.
On platforms that don't support Pthreads, e.g. IBM BlueGene/L or Cray XT3, the
default {\tt OFF} setting cannot be overridden!

\item[ESMF\_SITE]
Possible value: {\em site string}, {\tt default}

Build configure file site name or the value default. If not set, then the value
of default is assumed. When including platform-specific files, this value is
used as the third part of the directory name (parts 1 and 2 are the
ESMF\_OS value and ESMF\_COMPILER value, respectively.)

The Sourceforge {\tt esmfcontrib} repository contains makefiles which have
already been customized for certain machines.  If one exists for your site
and you wish to use it, download the corresponding files into the
{\tt build\_contrib} directory and set {\tt ESMF\_SITE} to your location
(which corresponds to the last part of the directory name).  See the
Sourceforge site
\htmladdnormallink{http://sourceforge.net/projects/esmfcontrib}{http://sourceforge.net/projects/esmfcontrib} for more information.

\item[ESMF\_TESTESMFMKFILE]
Possible value: {\tt ON}, {\tt OFF} (default)

Variable specifying whether the {\tt ESMFMKFILE} variable is evaluated to
determine which ESMF installation is being tested against. If set to the
value {\tt ON}, all tests and examples are build against the ESMF installation
referenced by the {\tt ESMFMKFILE} variable. For {\tt OFF}, the
{\tt ESMFMKFILE} variable is ignored and the tests and examples are build
against the ESMF under {\tt ESMF\_DIR}. This is the default.

\item[ESMF\_TESTEXHAUSTIVE]
Possible value: {\tt ON}, {\tt OFF} (default)

Variable specifying how to compile the unit tests. If set to the value {\tt ON},
then all unit tests will be compiled and will be executed when the test is
run.  If unset or set to any other value, only a subset of the unit tests
will be included to verify basic functions. Note that this is a compile-time
selection, not a run-time option.

\item[ESMF\_TESTFORCEOPENACC]
Possible value: {\tt ON}, {\tt OFF} (default)

The {\tt ON} setting enforces usage of OpenACC compiler flags when building ESMF test applications. This allows testing of user-level OpenACC usage even with {\tt ESMF\_OPENACC} set to {\tt OFF}.

\item[ESMF\_TESTFORCEOPENMP]
Possible value: {\tt ON}, {\tt OFF} (default)

The {\tt ON} setting enforces usage of OpenMP compiler flags when building ESMF test applications. This allows testing of user-level OpenMP usage even with {\tt ESMF\_OPENMP} set to {\tt OFF}.

\item[ESMF\_TESTHARNESS\_ARRAY]
Possible value: {\em test harness make target} (default not set)

Variable specifying the test harness makefile target for the array class.  If this variable is not specified, a default test scenario will be run for the array class.  See the ESMF Software Developer's Guide for instructions for selecting other test harness scenarios.

\item[ESMF\_TESTHARNESS\_FIELD]
Possible value: {\em test harness make target} (default not set)

Variable specifying the test harness makefile target for the field class.  If this variable is not specified, a default test scenario will be run for the field class.  See the ESMF Software Developer's Guide for instructions for selecting other test harness scenarios.

\item[ESMF\_TESTMPMD]
Possible value: {\tt ON}, {\tt OFF} (default)

Variable specifying whether to run MPMD-style tests, i.e. test applications
that start up as multiple separate executables.

\item[ESMF\_TESTSHAREDOBJ]
Possible value: {\tt ON}, {\tt OFF} (default)

Variable specifying whether to run shared object tests. This requires that the compute environment supports shared objects, and that the ESMF library is available in form of a shared library.

\item[ESMF\_TESTWITHTHREADS]
Possible value: {\tt ON}, {\tt OFF} (default)

If this environment variable is set to {\tt ON} {\em before} the ESMF system
tests are build they will activate ESMF threading in their code. Specifically
each component will be executed using ESMF single threading instead of the
default non-threaded mode. The difference between non-threaded and ESMF
single threaded execution should be completely transparent. Notice that the
setting of {\tt ESMF\_TESTWITHTHREADS} does {\em not} alter ESMF's dependency
on Pthreads but tests ESMF threading features during the system tests. An
ESMF library that was compiled with disabled Pthread features (via the {\tt
ESMF\_PTHREADS} variable) will produce ESMF error messages during system test
execution if the system tests were compiled with {\tt ESMF\_TESTWITHTHREADS}
set to {\tt ON}.

\item[ESMF\_TRACE\_LIB\_BUILD]
Possible value: {\em ON} (default), {\em OFF}

This variables determines whether extra libraries are built that are used
to add additional symbols to the ESMF tracing and profiling capability,
such as MPI communication functions.
If set to {\tt ON} the libraries are built and placed into the
{\tt ESMF\_INSTALL\_LIBDIR} alongside the ESMF library itself.

\item[ESMF\_XERCES]
See \ref{sec:xerces}

\item[ESMF\_XERCES\_INCLUDE]
See \ref{sec:xerces}

\item[ESMF\_XERCES\_LIBPATH]
See \ref{sec:xerces}

\item[ESMF\_XERCES\_LIBS]
See \ref{sec:xerces}

\item[ESMF\_YAMLCPP]
See \ref{sec:yaml-cpp}

\item[ESMF\_YAMLCPP\_INCLUDE]
See \ref{sec:yaml-cpp}

\item[ESMF\_YAMLCPP\_LIBPATH]
See \ref{sec:yaml-cpp}

\item[ESMF\_YAMLCPP\_LIBS]
See \ref{sec:yaml-cpp}

\end{description}

Environment variables must be set in the user's shell or when calling gmake. It
is {\em not} necessary to edit ESMF makefiles or other build system files to set
these variables. Here is an example of setting an environment variable in the
csh/tcsh shell:

\begin{verbatim}
  setenv ESMF_ABI 32
\end{verbatim}

In bash/ksh shell environment variables are set this way:

\begin{verbatim}
  export ESMF_ABI=32
\end{verbatim}

Environment variables can also be set from the gmake command line:

\begin{verbatim}
  gmake ESMF_ABI=32
\end{verbatim}

\subsection{Supported Platforms}
\input{../../build/doc/user_arch}

Building the library for multiple architectures or options at the same
time is supported; building or running the tests or examples is restricted
to one platform/architecture at a time.  The output from the test cases
will be stored in a separate directories so the results will be kept
separate for different architectures or options.

\subsection{Building the ESMF Library}
\label{BuildESMF}

% GNU make requirement.  File in build/doc
\input{../../build/doc/user_make}

Build the library with the command:
\begin{verbatim}
  gmake
\end{verbatim}

%Build options that enable you to copy the library and *.mod files to
%specified directories are explained in Section~\ref{BuildOptions}.

Makefiles throughout the framework are configured to allow users to
compile files only in the directory where {\tt gmake} is entered. Shared
libraries are rebuilt only if necessary. In addition the entire ESMF
framework may be built from any directory by entering {\tt gmake all},
assuming that all the environmental variables are set correctly as
described in Section~\ref{EnvironmentVariables}.

The makefiles are also configured to allow multiple make targets to be
compiled in parallel, via the gmake -j flag.  For example, to use eight
parallel processes to build the library, use -j8:
\begin{verbatim}
  gmake -j8 lib
\end{verbatim}

The parallel compilation feature depends on {\tt ESMF\_DEFER\_LIB\_BUILD=ON}
(the default) so that the library build will be deferred until all files
have been compiled.

The -j option should only be used during the creation of the library.
The test base and examples will not work correctly with -j set larger
than 1.

Users may also run examples or execute unit tests of specific classes
by changing directories to the desired class {\tt examples} or {\tt tests}
directories and entering {\tt gmake run\_examples} or
{\tt gmake run\_unit\_tests}, respectively.  For non-multiprocessor machines,
uni-processor targets are available as {\tt gmake run\_examples\_uni} or
{\tt gmake run\_unit\_tests\_uni}.

\subsection{Building the ESMF Documentation}
\label{BuildDocumentation}

The ESMF source documentation consists of an {\it ESMF User's Guide}
and an {\it ESMF Reference Manual for Fortran}.

The tarballs on the ESMF website for ESMF versions 3.0.1 and later do
not contain the ESMF documentation files.  The documentation is
available on the ESMF website in html or pdf form and most users should
not need to build it from the source.

If a user does want to build the documentation, they will need to download the
ESMF code repository (see section \ref{sec:download}). Latex, latex2html, perl
and csh must also be installed. For example, dependencies may be installed on
Ubuntu Linux using:
\begin{verbatim}
  [sudo] apt-get install texlive latex2html perl csh
\end{verbatim}

\noindent To build documentation:
\begin{verbatim}
  gmake doc              ! Builds the manuals, including pdf and html.
\end{verbatim}

\noindent The resulting documentation files will be
located in the top level directory \${ESMF\_DIR}/doc

\subsection{Installing the ESMF}
\label{InstallESMF}

The ESMF build system offers the standard {\tt install} target to install all
necessary files created during the build process into user specified locations.
The installation procedure will also install the ESMF documentation if it has
been built successfully following the procedure outlined above.

The installation location can be customized using six {\tt ESMF\_} environment
variables:
\begin{itemize}
\item {\tt ESMF\_INSTALL\_PREFIX} -- prefix for the other five variables.
\item {\tt ESMF\_INSTALL\_HEADERDIR} -- where to install header files.
\item {\tt ESMF\_INSTALL\_LIBDIR} -- where to install library files.
\item {\tt ESMF\_INSTALL\_MODDIR} -- where to install Fortran module files.
\item {\tt ESMF\_INSTALL\_BINDIR} -- where to install application files.
\item {\tt ESMF\_INSTALL\_DOCDIR} -- where to install documentation files.
\end{itemize}

Section~\ref{EnvironmentVariables} describes what each of these
environment variables does and how to set them.

Install ESMF with the command:
\begin{verbatim}
  gmake install
\end{verbatim}

Check the ESMF installation with the command:
\begin{verbatim}
  gmake installcheck
\end{verbatim}

{\em Advice to installers.} To complete the installation of ESMF, a single ESMF specific environment variable should be set. The variable is named {\tt ESMFMKFILE}, and it must point to the {\tt esmf.mk} file that was generated during the installation process. Systems that support multiple ESMF installations via management software (e.g. {\em modules, softenv, ...}) should set/reset the {\tt ESMFMKFILE} environment variable as part of the configuration.

Additionally, it is typically convenient to append the user's {\tt PATH} environment variable to provide access to the ESMF applications that were built during the installation process. The application binaries are located in the directory that was specified as {\tt ESMF\_INSTALL\_BINDIR} during the ESMF installation. The location is also stored in variable {\tt ESMF\_APPSDIR}, defined in file {\tt esmf.mk}. Systems that make ESMF installations available through management software (e.g. {\em modules, softenv, ...}) should modify the user's {\tt PATH} environment variable as part of the configuration.

{\em Hint.} By default, file {\tt esmf.mk} is located next to the ESMF library file in  directory {\tt ESMF\_INSTALL\_LIBDIR}. Consequently, unless {\tt esmf.mk} has been moved to a different location after the installation, the correct setting for {\tt ESMFMKFILE} is {\tt \$(ESMF\_INSTALL\_LIBDIR)/esmf.mk}.

{\em Rationale.} The only piece of information that is needed to use an ESMF installation is the exact location of the associated {\tt esmf.mk} file. This file contains all of the relevant settings and flags that allow a user to build their application against the ESMF installation. Standardizing the mechanism by which the location of {\tt esmf.mk} is made available to the user by the system will help users in the design of portable application build systems. (See sections \ref{sec:Use} and \ref{sec:CLTs} for details about the usage of {\tt esmf.mk}.) Further, modifying the user's {\tt PATH} environment variable is optional, since the location of the ESMF application binaries is available through the {\tt esmf.mk} file. However, setting the user's {\tt PATH} variable so that the ESMF applications are directly and conveniently accessible from the command line is recommended, especially if management software (e.g. {\em modules, softenv, ...}) is used on the system.
