\section{Project vision}
\label{sec:project_vision}

These are the broad objectives of the ESMF project:

\begin{itemize}
  
\item Facilitate the exchange of scientific codes so that researchers
  may more easily take advantage of the wealth of resources that are
  available in smaller-scale, process modeling and may more easily
  share experience among diverse large-scale modeling and data
  assimilation efforts.
  
\item Promote the reuse of standard technical software, the
  development of which now accounts for a substantial fraction of the
  software development budgets of large groups.  Any center developing
  or maintaining a large system for NWP, climate or seasonal
  prediction, data assimilation, or basic research now has to solve
  very similar software engineering and routine computational
  problems.
  
\item Focus community resources to deal with architectural changes and
  the lack of appropriate commodity middleware. The technical parts of
  the codes that would be dealt with in a common framework are also
  the most sensitive to architectural changes.
  
\item Present the computer industry with a unified, well-defined and
  well-documented task for them to address in their software design.
  The scientific community's influence with the industry might be much
  enhanced if exercised jointly by the major modeling centers.
  
\item Share the overhead costs of the software engineering aspects of
  model development: careful design, complete documentation, user
  training, and comprehensive testing. These are the efforts that are
  most easily neglected when corners have to be cut.
  
\item Provide institutional continuity to model and data assimilation
  development efforts.

\item Lay the foundation for developing standard classes, components, 
generic algorithms and data structures that can be used and customized
to solve a variety of problems, thereby increasing developer productivity.

\end{itemize}

These project objectives describe the desired impact of the ESMF on
the Earth system modeling community.  Although some of these
objectives are non-quantitative and difficult to verify, they still
must be satisfied to consider the ESMF project truly a success.  It is
essential to allow them to influence project execution, since it will
be possible to satisfy detailed functional requirements but fail to
achieve the underlying objectives of our effort.





